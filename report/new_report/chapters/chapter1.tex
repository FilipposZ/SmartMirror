\chapter{Εισαγωγή}
\label{chapter:intro}

\setlength{\parskip}{1em}
Τα τελευταία χρόνια τόσο η μείωση του κόστους παραγωγής όσο και η εξοικείωση του ευρέος κοινού με την τεχνολογία έχουν προωθήσει σημαντικά την ανάπτυξη και τη διάδοση των έξυπνων συσκευών. Οι έξυπνες συσκευές βρίσκουν εφαρμογή και αποδεικνύονται ολοένα και πιο χρήσιμες σε πολλαπλούς κλάδους. Από την περίθαλψη ασθενών στα νοσοκομεία, την αποφυγή κίνησης στους δρόμους (έξυπνη πόλη), τη βελτίωση των συστημάτων του αυτοκινήτου για καλύτερη οδηγική εμπειρία και αυξημένη ασφάλεια, μέχρι ακόμα και την κουζίνα ενός σπιτιού, οι συσκευές αυτές έχουν τη δυνατότητα να επιδράσουν θετικά στην ποιότητα ζωής μας. Συνδυάζοντας υλικό και λογισμικό εκτελούν εργασίες ή συλλέγουν και αναπαριστούν πληροφορίες, από τις οποίες επωφελούμαστε ατομικά και συλλογικά. 

Ειδικότερα οι έξυπνες συσκευές για το σπίτι έχουν προσελκύσει το ενδιαφέρον τόσο εταιρειών παραγωγής συσκευών για το σπίτι, οι οποίες επιθυμούν να αναβαθμίσουν τα προϊόντα τους, όσο και μηχανικών, οι οποίοι καινοτομούν δημιουργώντας τέτοιες συσκευές. Δικαιολογημένα συμβαίνει αυτό, εφόσον τέτοιες συσκευές μπορούν να συνεισφέρουν στην αυτοματοποίηση και κατά συνέπεια βελτιστιοποίηση καθημερινών εργασιών, βοηθώντας έτσι στην εξοικονόμηση χρόνου, καθώς και να προσφέρουν οι ίδιες εργαλεία για την οργάνωση χρόνου και σε αρκετές περιπτώσεις να λειτουργήσουν ως μέσο ψυχαγωγίας για τους χρήστες. Για παράδειγμα, ένα έξυπνος θερμοστάτης μπορεί να ρυθμίζει την θερμοκρασία ενός χώρου σε μια προκαθορισμένη από τον άνθρωπο τιμή δημιουργώντας ένα ευχάριστο περιβάλλον και εξοικονομώντας ενέργεια.

Μία οικιακή, κυρίως, συσκευή η οποία μπορεί να παρέχει μια μεγάλη γκάμα εφαρμογών στον χρήστη είναι ο έξυπνος καθρέφτης. Οι έξυπνοι καθρέφτες έρχονται να επαυξήσουν την ανακλαστική επιφάνεια του κλασσικού καθρέφτη με υλικό και λογισμικό, τα οποία θα παρέχουν πληροφορίες στον χρήστη είτε μέσω του διαδικτύου είτε μέσω αισθητήρων. Ίσως το πιο διαδεδομένο πρότζεκτ πάνω στον έξυπνο καθρέφτη αποτελεί το MagicMirror\footnote{\href{https://github.com/MichMich/MagicMirror}{https://github.com/MichMich/MagicMirror}} το οποίο είναι μία αρθρωτή (modular) πλατφόρμα πάνω στην οποία μπορεί κανείς να αναπτύξει και να εγκαταστήσει εφαρμογές για την επαύξηση των δυνατοτήτων του απλού καθρέφτη. Ένα ακόμη εγχείρημα, το οποίο έδωσε αρκετή έμπνευση και στην παρούσα διπλωματική, είναι o έξυπνος καθρέφτης\footnote{\href{https://github.com/HackerShackOfficial/Smart-Mirror}{https://github.com/HackerShackOfficial/Smart-Mirror}} της ομάδας Hacker Schack. Αξίζει επίσης να σημειωθεί ότι με μια απλή αναζήτηση στο διαδίκτυο συναντάει εύκολα κανείς και άλλες εφαρμογές πάνω στον έξυπνο καθρέφτη. Συμπεραίνουμε, επομένως, ότι ο έξυπνος καθρέφτης αποτελεί ελκυστικό αντικείμενο ενασχόλησης.

\section{Περιγραφή του Προβλήματος}
\label{section:problem_description}

Οι περισσότερες εφαρμογές που έχουν αναπτυχθεί πάνω στον έξυπνο καθρέφτη αυτή τη στιγμή αφορούν κυρίως στην ενημέρωση (news and social media feeds), στον συγχρονισμό ηλεκτρονικού ημερολογίου για λήψη υπενθυμίσεων και στην προβολή του καιρού αποτελώντας αρωγό σε καθημερινές δραστηριότητες. Όμως, οι απαιτήσεις χρηστών των παραπάνω εφαρμογών ικανοποιούνται ευκολότερα και αποτελεσματικότερα από άλλες συσκευές, όπως smartphone, καθιστώντας τον έξυπνο καθρέφτη μη ελκυστικό εμπορικό προϊόν. 

Βασική αιτία των παραπάνω αποτελεί το εγγενές πρόβλημα αλληλεπίδρασης με τον καθρέφτη, δηλαδή η έλλειψη αποδοτικών συσκευών εισόδου. Η επιλογή που φαντάζει λογικότερη ως είσοδο αποτελεί η φωνή του χρήστη δεδομένου ότι η επαφή του με τον καθρέφτη γίνεται από απόσταση. Όμως, στην ανίχνευση φωνής ελλοχεύουν δύο κύρια προβλήματα. Αρχικά, η αβεβαιότητα των συνθηκών του περιβάλλοντος στο οποίο τοποθετείται ο καθρέφτης δυσχεραίνει την ικανότητα του συστήματος να ανιχνεύει καθαρά τον ήχο. Επιπλέον, οι σημερινές τεχνολογίες αναγνώρισης φωνής και κατανόησης πρόθεσης του χρήστη επιδέχονται βελτίωση.

Επίσης, έως και σήμερα δεν έχει υλοποιηθεί κάποιο λειτουργικό σύστημα, το οποίο να χρησιμοποιείται καθολικά, κανονικοποιώντας έτσι τον τρόπο ανάπτυξης εφαρμογών πάνω στον έξυπνο καθρέφτη. Επομένως, οι λειτουργίες του κάθε συστήματος καθορίζονται από τον εκάστοτε κατασκευαστή μειώνοντας με αυτόν τον τρόπο την επεκτασιμότητα του έξυπνου καθρέφτη.

Παρόλα αυτά, ο έξυπνος καθρέφτης αποτελεί πρόσφορο έδαφος για την ανάπτυξη εφαρμογών που επαυξάνουν την βασική του λειτουργία, την προβολή του ειδώλου. Συνεχώς ο άνθρωπος αξιοποιεί έναν καθρέφτη προκειμένου να πάρει ανατροφοδότηση σχετικά με την εμφάνισή του, ώστε να συμπεράνει πόσο υγιής φαίνεται ή να αποκτήσει αυτοπεποίθηση, αλλά και σχετικά με την ορθότητα της στάσης και της κίνησής του όταν αθλείται, όπως για παράδειγμα στα γυμναστήρια. Ως εκ τούτου, ο έξυπνος καθρέφτης μπορεί να αποδειχθεί χρήσιμος συμπληρώνοντας τις εφαρμογές του απλού καθρέφτη στους τομείς της υγείας και της ευεξίας. Δυστυχώς, όμως, μέχρι και σήμερα δεν έχει αναπτυχθεί ικανός αριθμός εφαρμογών για τον έξυπνο καθρέφτη που αφορούν στην υγεία.


 
\section{Σκοπός - Συνεισφορά της Διπλωματικής Εργασίας}
\label{section:contribution}

Η παρούσα διπλωματική έχει δύο πρωταρχικούς στόχους, την ανάπτυξη ενός modular λειτουργικού συστήματος το οποίο θα επιτρέπει στους χρήστες να αλληλεπιδράσουν με τον έξυπνο καθρέφτη και την ανάπτυξη μιας εφαρμογής που αφορά στον έλεγχο της ορθότητας μιας άσκησης ενός αθλούμενου. Ο έλεγχος του καθρέφτη από τον χρήστη επιτυγχάνεται μέσω φωνητικών εντολών, ενώ ο έλεγχος της ορθότητας μέσω αποθηκευμένων ασκήσεων, καθορισμένες από ειδικούς, με τις οποίες συγκρίνεται η κίνηση του χρήστη.

Αρχικά, υλοποιήθηκε ένα ελαφρύ γραφικό περιβάλλον για την εμφάνιση πληροφοριών γύρω από το είδωλό του χρήστη. Στη συνέχεια, το λογισμικό εμπλουτίστηκε με λογική για ανάγνωση φωνητικών εντολών και αναγνώριση πρόθεσης με σκοπό την εκτέλεση των κατάλληλων πράξεων, όπως ανανέωση μιας ρύθμισης ή προβολή μιας νέας πληροφορίας. Τέλος, δόθηκε ιδιαίτερη μνεία στην επεκτασιμότητα του συστήματος, δηλαδή στην δυνατότητα εύκολης εγκατάστασης εξωτερικών εφαρμογών από τον χρήστη οι οποίες επαυξάνουν τις λειτουργίες του καθρέφτη.

Όσον αφορά τον δεύτερο στόχο της διπλωματικής, έγινε υλοποίηση μιας εφαρμογής για την εκτίμηση της ορθότητας μιας άσκησης. Η εφαρμογή αντιλαμβάνεται τον χρήστη μέσω κάμερας και κάνει εκτίμηση της πόζας του κατά τη διάρκεια εκτέλεσης της άσκησης. Τα δεδομένα της πόζας συγκρίνονται με ένα "ground truth", το οποίο ορίζεται από κάποιον ειδικό γυμναστή ή φυσιοθεραπευτή, και δίνεται ανατροφοδότηση στον χρήστη σε πραγματικό χρόνο σχετικά με την ακρίβεια εκτέλεσης της άσκησης. Επιπλέον, παρέχεται η δυνατότητα στον χρήστη να αποθηκεύσει την άσκησή του με σκοπό να την αναπαράξει ξανά στο μέλλον.
\section{Διάρθρωση της Αναφοράς}
\label{section:layout}

Η διάρθρωση της παρούσας διπλωματικής εργασίας είναι η εξής:

\begin{itemize}
	\item{\textbf{Κεφάλαιο \ref{chapter:sota}:} Γίνεται ανασκόπηση της ερευνητικής δραστηριότητας στον τομέα μέχρι σήμερα.
	}
	\item{\textbf{Κεφάλαιο \ref{chapter:theory}:} Αναλύεται το θεωρητικό υπόβαθρο. 
	}
	\item{\textbf{Κεφάλαιο \ref{chapter:tools}:} Παρουσιάζονται οι διάφορες τεχνικές και τα εργαλεία που χρησιμοποιήθηκαν στις υλοποιήσεις.
	}
	\item{\textbf{Κεφάλαιο \ref{chapter:implementations}:} Παρουσιάζονται τα βήματα της μεθοδολογίας που υλοποιήθηκε.
	}
	\item{\textbf{Κεφάλαιο \ref{chapter:examples}:} Περιγράφονται 3 παραδείγματα εφαρμογής των εργαλείων που αναπτύχθηκαν.
	}
	\item{\textbf{Κεφάλαιο \ref{chapter:conclusions}:} Παρουσιάζονται τα τελικά συμπεράσματα και προτείνονται θέματα για μελλοντική μελέτη, αλλαγές και επεκτάσεις.
	}

\end{itemize}


