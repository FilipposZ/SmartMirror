\section{Σκοπός - Συνεισφορά της Διπλωματικής Εργασίας}
\label{section:contribution}

Η παρούσα διπλωματική έχει ως στόχο την ανάπτυξη μιας μηχανής λογισμικού μοντελοστρεφούς λογικής, με την οποία οι χρήστες θα μπορούν να μοντελοποιούν συσκευές καθώς και την διασύνδεσή τους. Οι συσκευές αυτές θα μπορούν να είναι είτε μικροελεγκτές, είτε περιφερειακά (αισθητήρες και ενεργοποιητές), και όλα μαζί θα συνδέονται κατάλληλα για να συνθέσουν ένα σύστημα.

Αρχικά υλοποιήθηκαν δύο DSL για την περιγραφή των συσκευών και των μεταξύ τους συνδέσεων. Στην μία περιγράφονται τα χαρακτηριστικά των συσκευών (μνήμη, μονάδα επεξεργασίας, δικτύωση, ακροδέκτες κ.α.) και στην άλλη οι μεταξύ τους συνδέσεις (συνδέσεις ακροδεκτών, πρωτόκολλα επικοινωνίας που χρησιμοποιούνται κ.α.). Μέσω αυτών, δημιουργούνται τα κατάλληλα μοντέλα για τις συσκευές και συνδέσεις.

Από τα μοντέλα, πραγματοποιούνται δύο μετασχηματισμοί, ένας Model-to-Text (M2T) και ένας Model-to-Model (M2M). Ο M2M έχει ως αποτέλεσμα την παραγωγή διαγραμμάτων, τα οποία βοηθούν στην οπτικοποίηση της συνδεσμολογίας και ενδοεπικοινωνίας του συστήματος. Μέσω του M2T, παράγονται αυτόματα τμήματα λογισμικού που θα υλοποιούν κάποιες βασικές λειτουργίες (λήψη μετρήσεων από αισθητήρες, έλεγχος ενεργοποιητών κ.α.). Ο παραγόμενος κώδικας θα αφορά συσκευές που υποστηρίζονται από το λειτουργικό σύστημα πραγματικού χρόνου RIOT.