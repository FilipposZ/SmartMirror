\section{Περιγραφή του Προβλήματος}
\label{section:problem_description}


Πέρα από τα προβλήματα που αναλύθηκαν στην προηγούμενη παράγραφο, στα οποία δίνει λύση η MDE, ένα ακόμη σημαντικό θέμα που εμφανίζεται με την ανάπτυξη του κλάδου του IoT είναι η κατασκευή όλο και περισσότερων διαφορετικών IoT συσκευών. Φυσικά, λόγω αυτού από τη μία επεκτείνονται οι δυνατότητες που ένα IoT σύστημα μπορεί να έχει, από την άλλη όμως αυξάνεται η πολυπλοκότητα και ετερογένεια στο IoT.

Υπάρχει πληθώρα IoT συσκευών στην αγορά, όπως π.χ. τα έξυπνα ρολόγια, που διανέμονται έτοιμες για χρήση. Σε αυτές τις περιπτώσεις, οι χρήστες μπορούν να ακολουθήσουν σαφείς οδηγίες χρήσης από τον κατασκευαστή, και άρα πολύ εύκολα να αξιοποιήσουν τις δυνατότητες που η εκάστοτε συσκευή προσφέρει. Επομένως, το πρόβλημα της πολυπλοκότητας δεν εμφανίζεται σε τέτοιου είδους IoT συσκευές.

Στην περίπτωση όμως που κάποιος/α επιθυμεί να αναπτύξει ένα σύστημα με ΙοΤ συσκευές από την αρχή, επειδή π.χ. θέλει να πειραματιστεί ή να υλοποιήσει κάποιες λειτουργίες πιο εξειδικευμένες, τότε απαιτείται μια μεγάλη διαδικασία για την κατασκευή και άρα πολλές γνώσεις. Το πρώτο βήμα είναι η επιλογή των κατάλληλων μικροελεγκτών, αισθητήρων, ενεργοποιητών για την υλοποίηση της ιδέας. Απαιτείται λοιπόν γνώση πάνω στον τρόπο λειτουργίας των συσκευών αυτών, καθώς και στον τρόπο διασύνδεσης και επικοινωνίας τους. Ακολουθεί η ανάπτυξη λογισμικού για την υλοποίηση των επιθυμητών λειτουργιών, κάτι το οποίο από μόνο του σημαίνει πως πρέπει να υπάρχει εμπειρία με προγραμματισμό και πρωτόκολλα επικοινωνίας. Επίσης, σε πολλες περιπτώσεις, στο σύστημα που υλοποιείται απαιτείται η ύπαρξη ιδιοτήτων όπως η ακρίβεια στον χρόνο απόκρισης ή η χαμηλή κατανάλωση ενέργειας. Επομένως, απαιτούνται και οι γνώσεις των ιδιοτήτων των RTOS, καθώς και της κατάλληλης χρήσης τους.
