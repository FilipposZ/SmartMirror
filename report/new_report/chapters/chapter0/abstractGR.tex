\begin{center}
  \centering

  \vspace{0.5cm}
  \centering
  \textbf{\Large{Περίληψη}}
  \phantomsection
  \addcontentsline{toc}{section}{Περίληψη}

  \vspace{1cm}

\end{center}

Η μείωση του κόστους ενσωματωμένων συσκευών αλλά και η πρόοδος σχετικών τεχνολογιών του Internet of Things έχουν προκαλέσει μια έκρηξη στην ανάπτυξη έξυπνων συσκευών. Οι συσκευές αυτές αποδεικνύονται χρήσιμες για πλήθος ανθρώπων διότι έρχονται να αυτοματοποιήσουν τις εργασίες τους και να βελτιώσουν την ποιότητα ζωής τους.

Ο έξυπνος καθρέφτης αποτελεί μια τέτοιου είδους συσκευή, η οποία έρχεται να επαυξήσει τις δυνατότητες του κλασσικού καθρέφτη. Με την προσθήκη οθόνης πίσω από μια ανακλαστική, εν μέρη διαφανή, επιφάνεια είναι εφικτή η προβολή πληροφοριών παράλληλα με το είδωλο του ανθρώπου, ενώ με χρήση μικροϋπολογιστή και αισθητήρων ο καθρέφτης μπορεί να αποκτήσει λογική για την επίτευξη χρήσιμων λειτουργιών για τον χρήστη.

Η ανάπτυξη, όμως, εφαρμογών που αφορούν καθημερινές δραστηριότητες, όπως ενημέρωση, λήψη υπενθυμίσεων και συγχρονισμός ηλεκτρονικού ταχυδρομείου, δεν ικανοποιούνται ιδιαίτερα εύκολα από μια έξυπνη συσκευή όπως είναι ο καθρέφτης. Η ικανοποίηση των παραπάνω εφαρμογών καλύπτεται αποδοτικότερα και ευκολότερα από άλλες συσκευές όπως το κινητό ή ο υπολογιστής. Για τον λόγο αυτό μέχρι σήμερα ο έξυπνος καθρέφτης δεν αποτελεί ένα διαδεδομένο εμπορικό προϊόν.

Ένας τομέας, ωστόσο, που ο έξυπνος καθρέφτης μπορεί να αξιοποιηθεί είναι αυτός της υγείας και της ευεξίας. Το παραπάνω στηρίζεται στο γεγονός ότι ο άνθρωπος χρησιμοποιεί τον απλό καθρέφτη για να λάβει ανατροφοδότηση σχετικά με τον εαυτό του και την κατάσταση στην οποία βρίσκεται η υγεία του, λόγου χάρη βλέπει πόσο χλωμός είναι ή πόσο ορθά εκτελεί μία άσκηση. Επομένως, ο έξυπνος καθρέφτης μπορεί να εγκατασταθεί σε χώρους γυμναστηρίων, κλινικών αλλά και να λειτουργήσει ως ένας προσωπικός βοηθός υγείας στο σπίτι.

Τα δύο βασικά προβλήματα που γεννά η χρήση του έξυπνου καθρέφτη αποτελούν η ύπαρξη ενός ευρέως διαδεδομένου λειτουργικού συστήματος και η εύκολη αλληλεπίδραση του ανθρώπου μαζί του. Από την μία πλευρά, η υλοποίηση ενός καθολικά αποδεκτού λειτουργικού θα βοηθήσει στην εύκολη ανάπτυξη και επέκταση εφαρμογών πάνω στον έξυπνο καθρέφτη, με τον ίδιο τρόπο που συνέβαλλε το android στην επέκταση των έξυπνων κινητών τηλεφώνων. Από την άλλη πλευρά, η εύκολη χρήση και επικοινωνία με τον έξυπνο καθρέφτη μπορούν να συμβάλλουν στην εκτεταμένη υιοθέτηση του.

Η παρούσα διπλωματική υλοποιεί ένα ελαφρύ αρθρωτό λειτουργικό σύστημα πάνω στο οποίο είναι εφικτό να αναπτύξει κανείς εφαρμογές για τον έξυπνο καθρέφτη. Επιπρόσθετα, η αλληλεπίδραση μεταξύ χρήστη και έξυπνου καθρέφτη γίνεται με χρήση φωνητικών εντολών, τις οποίες αναγνωρίζει το σύστημα και εκτελεί τις επιθυμητές ενέργειες. Πέρα από αυτό, υλοποιήθηκε και μια εφαρμογή για τον έλεγχο της ορθότητας μια άσκησης η οποία βασίζεται σε τεχνολογία εκτίμησης πόζας. Τέλος, προτείνονται διάφορες επεκτάσεις και ιδέες για την περεταίρω ανάπυξη και βελτίωση του έξυπνου καθρέφτη.
