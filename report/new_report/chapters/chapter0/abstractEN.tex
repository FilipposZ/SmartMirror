{\fontfamily{cmr}\selectfont

\phantomsection
\addcontentsline{toc}{section}{Abstract}


\begin{center}
  \centering
  \textbf{\Large{Title}}
  \vspace{0.5cm}

  \textbf{\large{Implementation of interface for smart mirror with capabilities of poze estimation for personalized exercise and well-being applications}}

  \vspace{1cm}

  \centering
  \textbf{Abstract}
\end{center}


\begin{flushright}
  \vspace{2cm}
  Zacharopoulos Filippos \& Papageorgiou Dimitrios
  \\
  Signal Processing and Biomedical Technology Unit
  \\
  Electrical \& Computer Engineering Department,
  \\
  Aristotle University of Thessaloniki, Greece
  \\
  February 2022
\end{flushright}

}

The cost reduction of embedded devices as well as the progress in technologies pertaining the Internet of Things (\textit{IoT}) have caused a blast in smart devices development. These devices are proving to be useful for many people since they can automate their tasks and improve their quality of life.

The smart mirror constitutes such a smart device that comes to augment the capabilities of the classic mirror. By adding a monitor behind a reflective, partially transparent, surface it is feasible to show information in parallel with the human idol, while using a microprocessor and sensors the mirror can obtain logic for achieving useful operations for the user.

The development, however, of applications that concern daily activities such as news briefing, getting reminders and receiving emails are not very easily satisfied from a device such as the smart mirror. The gratification of the above tasks is covered more efficiently and easier by other devices like a smart phone or a computer. For this reason the smart mirror isn't commercially spread up to this date.

Nevertheless, the smart mirror can be utilized in the fields of health and well-being. The previous statement is backed by the fact that a human is using the classic mirror to receive feedback regarding himself and the status of their current health, for example they are watching how pale their skin is or how correct they execute on physical exercise. Therefore, the smart mirror can be installed in gym spaces, medical clinics and also operate like a personal health assistant at home.

The two major problems that arise from the use of the smart mirror are the lack of existence of a widely used operating system and the easy interaction between the human. On one hand, the implementation of a universally accepted operating system will help easy the development and extension of applications on top of the smart mirror, much like the same way that android has helped with the development of smart phones. On the other hand, easy usage and communication with the smart mirror can contribute in its extended adoption.

This diploma thesis implements a lightweight modular operating system on top of which one can develop applications for the smart mirror. Additionally, the interaction between human and smart mirror is achieved using voice commands, which the system is capable of recognizing and executing the desired actions. Apart from that, an application for controlling the correctness of a physical exercise was developed, which relies on pose estimation technology. Lastly, we propose various extensions and ideas for further development and improvement of the smart mirror.