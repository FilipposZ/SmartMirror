{\fontfamily{cmr}\selectfont

\phantomsection
\addcontentsline{toc}{section}{Abstract}


\begin{center}
  \centering
  \textbf{\Large{Title}}
  \vspace{0.5cm}

  %\textbf{\large{Simultaneous Localization \& Mapping Combining \\Particle Filters, Critical Rays Scan Match \& Topological Information}}
  \textbf{\large{Model-driven development for low-consumption real-time IOT devices}}

  \vspace{1cm}

  \centering
  \textbf{Abstract}
\end{center}


\begin{flushright}
  \vspace{2cm}
  Athanasios Manolis
  \\
  Intelligent Systems and Software Engineering Labgroup (ISSEL)
  \\
  Electrical \& Computer Engineering Department,
  \\
  Aristotle University of Thessaloniki, Greece
  \\
  September 2021
\end{flushright}

}

Internet of Things (\textit{IoT}) is a field that is evolving rapidly, especially in recent years. There is the possibility of developing even more applications which prove to be useful for many people, whether they have to do with simple functions in automation systems, or with larger scale applications in the industry. Therefore, more and more people want to work in this field.

The process of developing an IoT system involves code development to control the system's devices. In fact, in most cases fast response is of the utmost importance, so low-level code development is required, as well as the use of real-time operating systems (\textit{RTOS}). Also, due to the great heterogeneity of IoT devices on the market, it is necessary to understand the capabilities that each device can offer, in order to make the appropriate choice of one, tailored to the needs of the system to be implemented.

These requirements may seem complicated to some users, especially to people who are technologically untrained, i.e. do not have the necessary programming skills, but still want to build an IoT system e.g. for their personal use. This results in a large portion of people wanting to get involved with IoT, being discouraged to do so.

Model Driven Engineering (\textit{MDE}) is here to solve the problems that, those who want to get involved with IoT, may face, but also to simplify the software production process in general, as it can provide the developing of IoT systems to a more abstract level, which is more user friendly.

Through this diploma thesis, one is given the opportunity to describe, using models, IoT devices, through two domain specific Languages (\textit{DSL}) developed for the description of devices and the connections between them. From the models, a Model-to-Text (\textit{M2T}) transformation is performed for the automated code generation, for a variety of IoT devices, adapted to the characteristics that the user wishes for it to have. The software for controlling the IoT devices that is produced implements the process of taking measurements from sensors, and sending them to a \textit{broker}, but also the process of controlling actuators through the broker. It also consists of low-level code, as it has been designed according to the requirements of a real time operating system, named RIOT. Finally, a Model-to-Model (\textit{M2M}) transformation takes place in order to produce diagrams that provide a visualization and thus a better understanding by the user, of the wiring and intercommunication of their system.