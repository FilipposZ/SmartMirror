\section{Χειρισμός Φωνής}
\label{sec:voice}

Για την αλληλεπίδραση του χρήστη με τον καθρέφτη επιλέξαμε ως μέσο την φωνή του. Το κομμάτι της φωνής μπορεί να χωριστεί σε δύο μέρη, την είσοδο κατά την οποία θέλουμε να μετατρέψουμε τη φωνή σε κείμενο ούτως ώστε η εφαρμογή να καταλάβει την εντολή και να εκτελέσει την κατάλληλη ενέργεια (\textbf{Speech to Text}) και την έξοδο όπου σε αρκετές περιπτώσεις θα θέλαμε ο καθρέφτης να μιλάει στον χρήστη δίνοντας του πληροφορίες και κατά την οποία μετατρέπουμε το κείμενο σε φωνή (\textbf{Text to Speech}).

\subsection{SpeechRecognition}
Το SpeechRecognition\footnote{\href{https://github.com/Uberi/speech\_recognition}{https://github.com/Uberi/speech\_recognition}} είναι μια βιβλιοθήκη για την εκτέλεση αναγνώρισης φωνής όπου υποστηρίζει διάφορες μηχανές και APIs. Για τον καθρέφτη χρησιμοποιήθηκε το API της Google\footnote{\href{https://cloud.google.com/speech-to-text}{https://cloud.google.com/speech-to-text}}.

Για την λειτουργία της αναγνώρισης φωνής, απαιτείται η δημιουργία 2 αντικειμένων, ένα τύπου \texttt{Microphone} και ένα τύπου \texttt{Recognizer}. Το πρώτο αντικείμενο περιέχει μεθόδους για την είσοδο φωνής μέσω του μικροφώνου αλλά και παραμετροποίησης διαφόρων τιμών όπως η μέγιστη διάρκεια που θα ακούει το μικρόφωνο, τιμές σχετικές με τον θόρυβο φόντου κτλ, ενώ επιστρέφει ένα αντικείμενο που μπορεί να χρησιμοποιηθεί ως πηγή (source) στη συνέχεια. 

Το δεύτερο αντικείμενο, περιέχει μεθόδους για την ανάγνωση του ήχου και την μετατροπή του σε κείμενο μέσω του επιθυμητού υποστηριζόμενου API. Πιο συγκεκριμένα, μέσω της συνάρτησης \texttt{listen} μπορούμε να ηχογραφήσουμε από την πηγή (στον καθρέφτη η πηγή μας είναι το μικρόφωνο) και να αποθηκεύσουμε τα δεδομένα σε κατάλληλη μορφή για την αναγνώρισή τους αργότερα. Αυτό επιτυγχάνεται περιμένοντας το επίπεδο ενέργειας του ήχου να ξεπεράσει ένα κατώφλι (ο χρήστης ξεκίνησε να μιλάει) και σταματάει όταν η ενέργεια πέσει χαμηλά ή όταν δεν υπάρχει άλλη είσοδος ήχου. Η μέθοδος \texttt{listen()} περιέχει επίσης κάποιες χρήσιμες παραμέτρους για τον καθορισμό του μέγιστου χρόνου που θα περιμένει η εφαρμογή μέχρι ο χρήστης αρχίζει να μιλάει (\texttt{timeout}) και τον καθορισμό του μέγιστου χρόνου που θα περιμένει την φράση να τελειώσει (\texttt{phrase\_timeout}). Επομένως αν και οι 2 παράμετροι είναι ορισμένοι ο συνολικός χρόνος που θα ακούει η εφαρμογή είναι \texttt{timeout} + \texttt{phrase\_timeout}.

Τέλος, τα δεδομένα που επιστρέφονται από την συνάρτηση \texttt{listen()} δίνονται ως είσοδο στην κατάλληλη συνάρτηση αναγνώρισης για το επιθυμητό API. Στην εφαρμογή του καθρέφτη επιλέξαμε την μέθοδο \texttt{recognize\_google()} η οποία στέλνει τα δεδομένα στο API της Google και αυτό με τη σειρά του επιστρέφει σε κείμενο τα λόγια που ειπώθηκαν από τον χρήστη.

\subsection{Text To Speech}
Για την υλοποίηση της μετατροπής κειμένου σε φωνή χρησιμοποιήθηκαν δύο βιβλιοθήκες, η \texttt{gTTS}\footnote{\href{https://gtts.readthedocs.io/en/latest}{https://gtts.readthedocs.io/en/latest/}} και η \texttt{pydub}\footnote{\href{https://pydub.com/}{https://pydub.com/}}.

Η gTTS είναι μια βιβλιοθήκη που επιτρέπει την αλληλεπίδραση της εφαρμογής με το Text-To-Speech API\footnote{\href{https://cloud.google.com/text-to-speech}{https://cloud.google.com/text-to-speech}} της Google και αποθηκεύει τον ήχο ομιλίας σε ένα προσωρινό mp3 αρχείο. Στη συνέχεια, γίνεται χρήση της pydub βιβλιοθήκης για την αναπαραγωγή του ήχου στον χρήστη.