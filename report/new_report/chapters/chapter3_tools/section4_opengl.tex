
\section{OpenGL Shading Language ES}
\label{section:opengl_shading_language}

Στην εφαρμογή του \textsl{Exercisor} για την απεικόνιση του ανθρώπινου σώματος, όπου απαιτείται η προβολή σύνθετων τρισδιάστατων πλεγμάτων του ανθρώπινου σώματος \footnote{Η έξοδος του βαθειού νευρωνικού δικτύου της 3D εκτίμησης πόζας είναι ένα σύνολο κορυφών που σχηματίζουν το πλέγμα}, χρησιμοποιείται., χρησιμοποιούνται οι χαμηλότερου επιπέδου εντολές της OpenGL μέσω της διεπαφής προγραμματισμού που παρέχει το Kivy. Πιο συγκεκριμένα, για τον προσδιορισμό της θέσης, του χρώματος και του φωτισμού της κάθε κορυφής του πλέγματος στην οθόνη χρησιμοποιείται η γλώσσα προγραμματισμού OpenGL Shading Language για Ενσωματώμενα Συστήματα (GLSL for Embedded Systems, \textsl{GLSL ES}).

H \textsl{GLSL ES} είναι μία γλώσσα προγραμματισμού σκίασης υψηλού επιπέδου, αποτελόντας ένα υποσύνολο της GLSL, με παρόμοια σύνταξη με την γλώσσα προγραμματισμού C. Η γλώσσα διεπαφής (Application Programming Interface, \textsl{API}) της GLSL είναι ένα σύνολο συναρτήσεων, όμοιες με της C, και ένα σύνολο σταθερών ακεραίων (integer constants). Επιτρέπουν τον προγραμματισμό του καταχωρητή που αποδίδει τα πίξελ της οθόνης, τoυ \textsl{framebuffer} όπως αναφέρθηκε στην ενότητα \ref{section: opengl_framebuffer}. Tα επίπεδα επεξεργασίας της \textsl{OpenGL} είναι ο σκιαστής κορυφής (\textsl{vertex shader}) και ο σκιαστής τμήματος (\textsl{fragment shader}). Οι σκιαστές δέχονται μία λίστα μεταβλητών εισόδου και εξόδου και ορίζουν ομοιόμορφες μεταβλητές (\textsl{uniform variables}), οι οποίες υπολογίζονται κατά την εκτέλεση της εκάστοτε συνάρτησης σκίασης. Η επικοινωνία ανάμεσα στους σκιαστές επιτυγχάνεται ορίζοντας ίδια ονόματα μεταβλητών. Ο \textsl{vertex shader} δέχεται ως είσοδο τη θέση και τα τα κάθετα διανύσματα, (\textsl{normals} βλ. παράρτημα \ref{normals}) για κάθε κορυφή η οποία αποθηκεύεται στον framebuffer, όπως περιγράφηκε στην ενότητα \ref{section: opengl_framebuffer}. 