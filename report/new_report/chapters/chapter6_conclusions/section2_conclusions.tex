\section{Συμπεράσματα}
\label{sec:conclusions}

Η παρούσα διπλωματική εργασία υλοποιεί ένα ελαφρύ, αρθρωτό λειτουργικό σύστημα για χρήση σε έξυπνους καθρέφτες με στόχο την εύκολη ανάπτυξη εφαρμογών που αφορούν τον τομέα της υγείας. Το γραφικό περιβάλλον του λειτουργικού είναι σχεδιασμένο με τρόπο που επαυξάνει την απλή ανακλαστική επιφάνεια του απλού καθρέφτη χαρίζοντας την δυνατότητα στον χρήστη για ανάγνωση πληροφορίων ενώ ταυτόχρονα παραμένει ορατό το είδωλό του. Επίσης, η αλληλεπίδραση μεταξύ χρήστη και συστήματος επιτυγχάνεται μέσω φωνητικών εντολών, οι οποίες αναγνωρίζονται από τον έξυπνο καθρέφτη που με τη σειρά του εκτελεί κατάλληλες επιθυμητές ενέργειες. Τέλος, ιδιαίτερη μνεία έχει δοθεί στην επεκτασιμότητα του συστήματος προκειμένου να μπορεί ο καθένας να αναπτύξει χωρίς μεγάλη δυσκολία την δική του εφαρμογή πάνω στον έξυπνο καθρέφτη.

Ταυτόχρονα, αναπτύχθηκε και ενσωματώθηκε μία εφαρμογή για την εκτίμηση της ορθότητας μιας άσκησης. Ο χρήστης θα μπορεί να αθλείται μπροστά στον έξυπνο καθρέφτη και να λαμβάνει ανατροφοδότηση σχετικά με την εκτελούμενη άσκηση αλλά και θα μπορεί να αποθηκεύει προηγούμενες ασκήσεις για αναπαραγωγή στο μέλλον. Τα αποθηκευμένα δεδομένα βάση των οποίων ελέγχεται η ορθότητα της εκτελούμενης άσκησης μπορούν να παρέχονται από οποιονδήποτε ειδικό ιατρό ή φυσιοθεραπευτή διευρύνοντας έτσι την γκάμα των διαθέσιμων ασκήσεων.

Το λογισμικό ήταν αρχικά σχεδιασμένο να εκτελεστεί πάνω στο Raspberry Pi 4 Model B\footnote{\href{https://www.raspberrypi.com/products/raspberry-pi-4-model-b/}{https://www.raspberrypi.com/products/raspberry-pi-4-model-b/}}, κάτι που δεν πέτυχε, αφού εξαιτίας των απαιτήσεων της εφαρμογής για τον έλεγχο της άσκησης η οποία χρησιμοποιεί νευρωνικά δίκτυα δεν ήταν εφικτή η ομαλή λειτουργία του συστήματος.
