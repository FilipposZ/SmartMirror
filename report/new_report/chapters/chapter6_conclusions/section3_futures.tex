\section{Μελλοντικές επεκτάσεις}
\label{sec:futures}

Μια λύση στο πρόβλημα επιδόσεων του Raspberry Pi θα ήταν η ενσωμάτωση του λογισμικού στο Nvidia Jetson Nano\footnote{\href{https://developer.nvidia.com/embedded/jetson-nano}{https://developer.nvidia.com/embedded/jetson-nano}} το οποίο έχει αυξημένες δυνατότητες, λόγω της κάρτας γραφικών, για εκτέλεση αλγορίθμων μηχανικής μάθησης. Επίσης, ο κώδικας του Controller του λειτουργικού συστήματος χρήζει βελτίωσης για καλύτερο έλεγχο των εγκατεστημένων εφαρμογών, όπως παραλληλοποίηση των εργασιών μέσω threads, αλλά και μεγαλύτερη εξοικονόμηση ενέργειας σε περιόδους αδράνειας του έξυπνου καθρέφτη.

Όσον αφορά την εφαρμογή εκτίμησης πόζας, προτείνονται δύο τομείς για βελτίωση ή επέκταση. Από την μία πλευρά, η απεικόνιση του μοντέλου του χρήστη μπορεί να ωραιοποιηθεί ικανοποιώντας καλύτερα τις αισθητικές απαιτήσεις του. Προς επίτευξη αυτού του σκοπού μπορούν να αξιοποιηθούν οι δυνατότητες της OpenGL με σκοπό την ομορφότερη απεικόνιση του πλέγματος του ανθρώπινου μοντέλου ή να ενσωματωθούν μοντέλα μηχανικής μάθησης όπως το pix2surf\footnote{\href{https://github.com/aymenmir1/pix2surf}{https://github.com/aymenmir1/pix2surf}} εξατομικεύοντας την πρόσοψη του χρήστη. Από την άλλη πλευρά, η λειτουργία επεξεργασίας των δεδομένων των ασκήσεων αναφοράς θα μπορούσε να λειτουργήσει ως ένα framework σχολιασμού δεδομένων για εκπαίδευση αντίστοιχων μοντέλων μηχανικής μάθησης.

Επιπλέον, οι δυνατότητες του καθρέφτη μπορούν να επαυξηθούν μελλοντικά με προσάρτηση νέων εφαρμογών που αφορούν την υγεία, αφού ο έξυπνος καθρέφτης είναι αρκετά βολικότερος για την λήψη βιομετρικών μετρήσεων σε σχέση με την επίσκεψη σε μια κλινική ή ένα νοσοκομείο. Στο \cite{reflect_health_paper} παρουσιάζονται αρκετοί τομείς της υγείας στους οποίους ο έξυπνος καθρέφτης μπορεί να βοηθήσει τους ασθενείς. Η αναγνώριση συναισθημάτων, η εκτίμηση ρίσκου για καρδιαγγειακά νοσήματα, η λήψη μετρικών σχετικά με το αίμα όπως το ζάχαρο και η μέτρηση καρδιακών παλμών αποτελούν ενδεικτικές εφαρμογές για μελλοντική έρευνα και ανάπτυξη.

Εντούτοις, για την προσάρτηση νέων εφαρμογών θα ήταν ιδιαιτέρως βολική η ύπαρξη ενός κεντρικού διακομιστή ο οποίος θα συμβάλλει στην εύρωστη και εύχρηστη λειτουργία του έξυπνου καθρέφτη. Έτσι, θα γίνει εφικτή η αυτοματοποίηση εγκατάστασης και αναβάθμισης εξωτερικών εφαρμογών και η λήψη δεδομένων που συμπληρώνουν τη λειτουργία τους. Για παράδειγμα, στην υπάρχουσα εφαρμογή εκτίμησης πόζας θα δίνεται η δυνατότητα σε έναν ειδικό ιατρό ή φυσιοθεραπευτή να παρέχει απευθείας τις ασκήσεις αναφοράς.

Κλείνοντας, παρά την δεδομένη αξία που μπορεί να προσδώσει ο έξυπνος καθρέφτης στους τομείς της υγείας και της ευεξίας, εξακολουθούν να χρειάζονται ακόμη αρκετές προσπάθειες για την ευρεία υιοθέτησή του. Η ανάπτυξη ενός ενοποιημένου λειτουργικού συστήματος στο οποίο θα μπορούν όλοι να έχουν πρόσβαση, η συνεχής πρόοδος των αλγορίθμων μηχανικής μάθησης που θα επιτρέπουν την λήψη αξιόπιστων αποτελεσμάτων αλλά και η βελτιστοποίηση της αναγνώρισης φωνής για την ευκολότερη αλληλεπίδραση μεταξύ του καθρέφτη είναι κάποιοι από τους παράγοντες που θα βοηθήσουν στην διάδοση του έξυπνου καθρέφτη τόσο στις κλινικές όσο και στα σπίτια.