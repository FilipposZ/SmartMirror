\section{Μελλοντικές επεκτάσεις}
\label{sec:futures}

Μια λύση στο πρόβλημα επιδόσεων του Raspberry Pi θα ήταν η ενσωμάτωση του λογισμικού στο Nvidia Jetson Nano\footnote{\href{https://developer.nvidia.com/embedded/jetson-nano}{https://developer.nvidia.com/embedded/jetson-nano}} το οποίο έχει αυξημένες δυνατότητες για εκτέλεση αλγορίθμων μηχανικής μάθησης. Επίσης, ο κώδικας του Controller του λειτουργικού συστήματος χρήζει βελτίωσης για καλύτερο έλεγχο των εγκατεστημένων εφαρμογών, όπως παραλληλοποίηση των εργασιών, και αποδοτικότερη αξιοποίηση του υλικού του συστήματος.

Επιπλέον, οι δυνατότητες του καθρέφτη μπορούν να επαυξηθούν μελλοντικά με προσάρτηση νέων εφαρμογών που αφορούν την υγεία, αφού ο έξυπνος καθρέφτης είναι αρκετά βολικότερος για την λήψη μετρήσεων σε σχέση με την επίσκεψη σε μια κλινική ή ένα νοσοκομείο. Στο \cite{reflect_health_paper} παρουσιάζονται αρκετοί τομείς της υγείας στους οποίους ο έξυπνος καθρέφτης μπορεί να βοηθήσει τους ασθενείς. Η αναγνώριση συναισθημάτων, η εκτίμηση ρίσκου για καρδιαγγειακά νοσήματα, η λήψη μετρικών σχετικά με το αίμα όπως το ζάχαρο και η μέτρηση καρδιακών παλμών αποτελούν ενδεικτικές εφαρμογές για μελλοντική έρευνα και ανάπτυξη.

Κλείνοντας, παρά την δεδομένη αξία που μπορεί να προσδώσει ο έξυπνος καθρέφτης στους τομείς της υγείας και της ευεξίας, εξακολουθούν να χρειάζονται ακόμη αρκετές προσπάθειες για την ευρεία υιοθέτησή του. Η ανάπτυξη ενός ενοποιημένου λειτουργικού συστήματος στο οποίο θα μπορούν όλοι να έχουν πρόσβαση, η συνεχής πρόοδος των αλγορίθμων μηχανικής μάθησης που θα επιτρέπουν την λήψη αξιόπιστων αποτελεσμάτων αλλά και η βελτιστοποίηση της αναγνώρισης φωνής για την ευκολότερη αλληλεπίδραση μεταξύ του καθρέφτη είναι κάποιοι από τους παράγοντες που θα βοηθήσουν στην διάδοση του έξυπνου καθρέφτη τόσο στις κλινικές όσο και στα σπίτια.