\chapter{Συμπεράσματα και Μελλοντικές επεκτάσεις}
\label{chapter:conclusions}

\section{Συμπεράσματα}

Η παρούσα διπλωματική εργασία παρουσιάζει μια DSL που παρέχει τη μοντελοποίηση συσκευών και της μεταξύ τους επικοινωνίας σε IoT συστήματα, που χρησιμοποιούν το λειτουργικό RIOT. Επίσης, παρέχει στον χρήστη τη δυνατότητα να παράξει αυτόματα κώδικα προς εκτέλεση, για συγκεκριμένους μικροελεγκτές και περιφερειακά, προσαρμοσμένο στις παραμέτρους που δίνει στα μοντέλα του. Ο εκτελέσιμος κώδικας, εκτελεί κάποιες βασικές λειτουργίες που μπορεί να χρειαστούν σε ένα IoT σύστημα, όπως η λήψη μετρήσεων από αισθητήρες και ο έλεγχος ενεργοποιητών μέσω δημοσίευσης μηνυμάτων σε κάποιον broker.

Κατ' αυτόν τον τρόπο, ο χρόστης μπορεί να κατασκευάσει σε ένα πιο αφαιρετικό επίπεδο το IoT σύστημα που επιθυμεί, χωρίς να χρειάζεται να είναι πλήρως τεχνολογικά καταρτισμένος στον χαμηλού επιπέδου κώδικα που απαιτείται για εφαρμογές σε λειτουργικά συστήματα πραγματικού χρόνου, όπως και αυτό που χρησιμποιείται στην παρούσα εργασία, το RIOT.

Τα διαγράμματα που επίσης παράγονται, μπορούν να δώσουν στον χρήστη μια καλύτερη αντίληψη του συστήματος που θέλει να δημιουργήσει, όσον αφορά τη συνδεσμολογία και τον τρόπο επικοινωνίας μεταξύ των συσκευών.

Τέλος, βάσει των κανόνων της DSL, ο χρήστης μπορεί εύκολα και έγκαιρα να αντιληφθεί πιθανά λάθη που έχει στο σύστημά του, και άρα να γλιτώσει χρόνο στη δημιουργία του.

\section{Μελλοντικές επεκτάσεις}

Οι δυνατότητες που παρέχονται από την παρούσα εργασία, είναι σε αρκετά αρχικό στάδιο ενός IoT συστήματος. Επομένως, μια πολύ ενδιαφέρουσα επέκταση θα ήταν να συγγραφούν πρότυπα αρχεία κώδικα για περισσότερες λειτουργίες του εκάστοτε περιφερειακού, και άρα χρήστης να έχει τη δυνατότητα επιλογής.

Επίσης, όπως ήδη αναφέρθηκε, τα εργαλεία που αναπτύχθηκαν υποστηρίζουν μόνο ένα συγκεκριμένο αριθμό μικροελεγκτών και περιφερειακών και ένα λειτουργικό σύστημα, το RIOT. Στον κόσμο του ΙοΤ υπάρχει πληθώρα συσκευών και λειτουργικών συστημάτων που το καθένα προσφέρει μοναδικές λειτουργίες οι οποίες θα μπορούσα να είναι χρήσιμες στους χρήστης. Μια πολλή καλή προσθήκη λοιπόν, θα ήταν το εργαλείο αυτό να δίνει τη δυνατότητα στον χρήστη να χρησιμοποιήσει περισσότερες συσκευές στο σύστημά του, ή ακόμα και να παράγεται κώδικας για πληθώρα λειτουργικών συστημάτων.

Τέλος, ο τρόπος με τον οποίο περιγράφονται τα μοντέλα από τον χρήστη βασίζεται σε αναπαραστάσεις κειμένου (textual language/representation), κάτι το οποίο θα μπορούσε να είναι αποθαρρυντικό για κάποιους χρήστες. Μια πολύ σημαντική επέκταση θα ήταν η ανάπτυξη γραφικού περιβάλλοντος για την σχεδίαση των μοντέλων, ακολουθώντας τεχικές μοντελοστραφούς ανάπτυξης με χρήση γραφικών αναπαραστάσεων.