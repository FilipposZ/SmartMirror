\section{Απαιτήσεις Συστήματος}
\label{sec:system_requirements}


Οι απαιτήσεις συστήματος εκφράζουν τις ανάγκες και τους περιορισμούς που τίθενται σε ένα σύστημα το οποίο προσπαθεί να λύσει ένα πραγματικό πρόβλημα \cite{swebok}.

Οι απαιτήσεις από το σύστημα χωρίζονται σε δύο κατηγορίες, τις λειτουργικές (\textbf{ΛΑ}) και τις μη λειτουργικές (\textbf{ΜΛΑ}) απαιτήσεις. Οι λειτουργικές απαιτήσεις αντανακλούν τη δυνατότητα εκτέλεσης ενεργειών που παρέχει το σύστημα στον χρήστη, ενώ οι μη λειτουργικές απαιτήσεις είναι αυτές που χαρακτηρίζουν το σύστημα και προσδιορίζουν ποιοτικά χαρακτηριστικά στις λειτουργίες του. Μπορεί, επίσης, να τεθούν περιορισμοί και στόχοι ως προς τη συμπεριφορά του συστήματος, όπως και απαιτήσεις ως προς την εξελιξιμότητά του.

\subsection{Λειτουργικές Απαιτήσεις}
\subsubsection{\underline{ΛΑ-1}}
\noindent Ο χρήστης πρέπει να μπορεί να δίνει φωνητικές εντολές.

\noindent\textbf{Περιγραφή:} Ο χρήστης πρέπει να μπορεί να δίνει φωνητικές εντολής μέσω της φωνής του τις οποίες ο καθρέφτης θα εκτελεί.

\noindent\textbf{User Priority (5/5):} Η δυνατότητα φωνητικών εντολών είναι πάρα πολύ σημαντική για τον χρήστη αφού θα είναι ο κύριος τρόπος επικοινωνίας και αλληλεπίδρασής του με τον καθρέφτη.

\noindent\textbf{Technical Priority (5/5):} Η δυνατότητα φωνητικών εντολών είναι υψίστης σημασίας για το σύστημα, καθώς πρέπει να είναι σε θέση να αναγνωρίζει τις προθέσεις του χρήστη και να εκτελεί τις κατάλληλες ενέργειες.

\subsubsection{\underline{ΛΑ-2}}
\noindent Ο χρήστης πρέπει να μπορεί να ανοίγει διαφορετικές εφαρμογές.

\noindent\textbf{Περιγραφή:} Ο χρήστης πρέπει να μπορεί να χρησιμοποιεί τις διαφορετικές εφαρμογές που είναι εγκατεστημένες στον καθρέφτη.

\noindent\textbf{User Priority(5/5):} Η δυνατότητα χρήστης διαφορετικών εφαρμογών είναι πολύ σημαντική για τον χρήστη, καθώς με αυτόν τον τρόπο μπορεί να εκτελέσει τις διαφορετικές λειτουργίες που του παρέχει ο καθρέφτης

\noindent\textbf{Technical Priority (5/5):} Η δυνατότητα χρήσης διαφορετικών εφαρμογών είναι πολύ σημαντική για το σύστημα, αφού θα πρέπει να είναι σε θέση να εκτελεί τις διαφορετικές εφαρμογές που είναι εγκατεστημένες στον καθρέφτη.

\subsubsection{\underline{ΛΑ-3}}
\noindent Ο χρήστης πρέπει να μπορεί να εγκαθιστά διαφορετικές εφαρμογές.

\noindent\textbf{Περιγραφή:} Ο χρήστης πρέπει να μπορεί να εγκαθιστά στον καθρέφτη εφαρμογές που αναπτύσσει αυτός ή κάποιος τρίτος.

\noindent\textbf{User Priority (4/5):} Η δυνατότητα εγκατάστασης διαφορετικών εφαρμογών είναι αρκετά σημαντική για τον χρήστη, καθώς με αυτόν τον τρόπο θα μπορεί να επεκτείνει τις δυνατότητες του καθρέφτη.

\noindent\textbf{Technical Priority (5/5):} Η δυνατότητα εγκατάστασης διαφορετικών εφαρμογών είναι πολύ σημαντική για το σύστημα, αφού θα πρέπει να δέχεται εύκολα εξωτερικές εφαρμογές και να τις εκτελεί χωρίς προβλήματα. 

\subsubsection{\underline{ΛΑ-4}}
\noindent Ο χρήστης πρέπει να μπορεί να ενεργοποιεί και να απενεργοποιεί τον καθρέφτη με φωνητικές εντολές.

\noindent\textbf{Περιγραφή:} Ο χρήστης πρέπει να μπορεί να ενεργοποιεί και να απενεργοποιεί τον καθρέφτη δίνοντάς του κατάλληλες εντολές ανοίγματος/κλεισίματος.

\noindent\textbf{User Priority (5/5):} Η δυνατότητα ενεργοποίησης/απενεργοποίησης του καθρέφτη με εντολές είναι πολύ σημαντική για τον χρήστη, καθώς θα μπορεί να ελέγξει πότε ο καθρέφτης να είναι σε λειτουργία

\noindent\textbf{Technical Priority (5/5):} Η δυνατότητα ενεργοποίησης/απενεργοποίησης του καθρέφτη με εντολές είναι πολύ σημαντική για το σύστημα, καθώς θα μπορεί να ελεγχθεί η κατάσταση λειτουργίας του.

\subsubsection{\underline{ΛΑ-5}}
\noindent Ο χρήστης πρέπει να μπορεί να αλλάζει τις ρυθμίσεις του καθρέφτη.

\noindent\textbf{Περιγραφή:} Ο χρήστης πρέπει να μπορεί να αλλάζει τις ρυθμίσεις του καθρέφτη είτε μέσω εντολών είτε μέσω κειμένου.

\noindent\textbf{User Priority (5/5):} Η δυνατότητα αλλαγής των ρυθμίσεων είναι πολύ σημαντική για τον χρήστη, καθώς θα μπορεί να εξατομικεύσει τις λειτουργίες του καθρέφτη ανάλογα με τα θέλω του.

\noindent\textbf{Technical Priority (5/5):} Η δυνατότητα αλλαγής των ρυθμίσεων είναι πολύ σημαντική για το σύστημα, καθώς θα είναι εφικτή η ορθή και εξατομικευμένη λειτουργία των εφαρμογών του καθρέφτη.

\subsubsection{\underline{ΛΑ-6}}
\noindent Ο χρήστης πρέπει να μπορεί να δει την ημερομηνία και ώρα.

\noindent\textbf{Περιγραφή:} Ο χρήστης πρέπει να μπορεί να δει την τωρινή ημερομηνία και την ώρα στην αρχική οθόνη του καθρέφτη

\noindent\textbf{User Priority (5/5):} Η δυνατότητα ενημέρωσης για την ημερομηνία και ώρα είναι πολύ σημαντική για τον χρήστη προκειμένου να μπορεί να οργανώσει καλύτερα το πρόγραμμά του.

\noindent\textbf{Technical Priority (5/5):} Η δυνατότητα ενημέρωσης για την ημερομηνία και ώρα είναι πολύ σημαντική για το σύστημα αφού με βάση αυτά μπορεί να επηρεαστεί η λειτουργία του.

\subsubsection{\underline{ΛΑ-7}}
\noindent Ο χρήστης πρέπει να μπορεί να δει τον καιρό σε ζωντανό χρόνο.

\noindent\textbf{Περιγραφή:} Ο χρήστης πρέπει να μπορεί να δει τον καιρό σε ζωντανό χρόνο στην αρχική οθόνη του καθρέφτη.

\noindent\textbf{User Priority (5/5):} Η δυνατότητα ζωντανής ενημέρωσης του καιρού είναι πολύ σημαντική για τον χρήστη, καθώς μπορεί να οργανώσει καλύτερα το πρόγραμμά του.

\noindent\textbf{Technical Priority (4/5):} Η δυνατότητα ζωντανής ενημέρωσης του καιρού είναι αρκετά σημαντική για το σύστημα αφού μπορεί να χρησιμοποιήσει τις πληροφορίες για βελτίωση της λειτουργίας του

\subsubsection{\underline{ΛΑ-8}}
\noindent Ο χρήστης πρέπει να μπορεί να κάνει ερωτήσεις σχετικά με τον καιρό.

\noindent\textbf{Περιγραφή:} Ο χρήστης πρέπει να μπορεί να κάνει ερωτήσεις και να ενημερώνεται σχετικά με τον καιρό.

\noindent\textbf{User Priority (4/5):} Η δυνατότητα ερωτήσεων σχετικά με τον καιρό είναι αρκετά σημαντική για τον χρήστη αφού μπορεί να ενημερωθεί και να οργανώσει καλύτερα το πρόγραμμά του.

\noindent\textbf{Technical Priority (2/5):} Η δυνατότητα ερωτήσεων σχετικά με τον καιρό δεν είναι πολύ σημαντική για την εύρυθμη λειτουργία του συστήματος αφού μπορεί να λειτουργήσει ομαλά και χωρίς αυτήν.


\noindent\subsection{Μη Λειτουργικές Απαιτήσεις}

\subsubsection{\underline{ΜΛΑ-1}}
\noindent Το σύστημα πρέπει να αποκρίνεται στις εντολές του χρήστη σε λιγότερο από 500ms.

\noindent\textbf{Περιγραφή:} Το σύστημα πρέπει να αναγνωρίζει τις προθέσεις του χρήστη και να ανταποκρίνεται κατάλληλα σε λιγότερο από 500ms.

\noindent\textbf{User Priority (5/5):} Η απαίτηση αυτή είναι πολύ σημαντική για τον χρήστη, καθώς θέλει ο καθρέφτης να του προσφέρει μια διαδραστική εμπειρία πραγματικού χρόνου.

\noindent\textbf{Technical Priority (5/5):} Η απαίτηση αυτή είναι πολύ σημαντική για το σύστημα, καθώς θα καθορίσει τα τεχνικά χαρακτηριστικά και τη δομή του, προκειμένου να καλύπτεται η απαιτούμενη ταχύτητα απόκρισης (500ms).

\subsubsection{\underline{ΜΛΑ-2}}
\noindent Το σύστημα πρέπει να ανταποκρίνεται ορθά όταν η εντολή του χρήστη δεν αναγνωρίζεται

\noindent\textbf{Περιγραφή:} Το σύστημα πρέπει να χειρίζεται εντολές του χρήστη που δεν αναγνωρίζονται και να μην καταρρέει.

\noindent\textbf{User Priority (5/5):} Η απαίτηση αυτή είναι πολύ σημαντική για τον χρήστη, καθώς θέλει ο καθρέφτης να λειτουργεί χωρίς σφάλματα.

\noindent\textbf{Technical Priority (5/5):} Η απαίτηση αυτή είναι πολύ σημαντική για το σύστημα, καθώς ο χειρισμός των "κακών" εισόδων θα βοηθήσει τον καθρέφτη να λειτουργεί ορθά χωρίς σφάλματα.

\subsubsection{\underline{ΜΛΑ-3}}
\noindent Το σύστημα πρέπει να επιτρέπει την προσάρτηση εφαρμογών.

\noindent\textbf{Περιγραφή:} Το σύστημα πρέπει να επιτρέπει την προσάρτηση εφαρμογών και την εύκολη ενσωμάτωσή τους στην ροή του προγράμματος.

\noindent\textbf{User Priority (5/5):} Η απαίτηση αυτή είναι πολύ σημαντική για τον χρήστη, καθώς η ενσωμάτωση εξωτερικών εφαρμογών θα επεκτείνει τις δυνατότητες που θα του δίνει ο καθρέφτης.

\noindent\textbf{Technical Priority (5/5):} Η απαίτηση αυτή είναι πολύ σημαντική για το σύστημα, καθώς θα πρέπει να σχεδιαστεί με τέτοιο τρόπο ώστε να είναι εύκολη η επεκτασιμότητα των εφαρμογών του καθρέφτη.

\subsubsection{\underline{ΜΛΑ-4}}
\noindent Το σύστημα πρέπει να επιτρέπει την ταυτόχρονη εμφάνιση εφαρμογών και του ειδώλου του χρήστη.

\noindent\textbf{Περιγραφή:} Το σύστημα πρέπει να έχει ανακλαστική οθόνη στην οποία ο χρήστης να μπορεί να δει τόσο τις εφαρμογές του καθρέφτη όσο και το είδωλό του.

\noindent\textbf{User Priority (5/5):} Η απαίτηση αυτή είναι πολύ σημαντική για τον χρήστη, καθώς θέλει να μπορεί να αλληλεπιδρά με τον καθρέφτη καθώς βλέπει το είδωλό του.

\noindent\textbf{Technical Priority (5/5):} Η απαίτηση αυτήν είναι πολύ σημαντική για το σύστημα, καθώς θα καθορίσει το είδος της οθόνης που θα χρησιμοποιηθεί και την τοποθέτηση των εφαρμογών πάνω σε αυτήν.

\subsubsection{\underline{ΜΛΑ-5}}
\noindent Το σύστημα πρέπει να υποστηρίζει σύνδεση με το Wit.ai.

\noindent\textbf{Περιγραφή:} Το σύστημα πρέπει να υποστηρίζει σύνδεση με το Wit.ai προκειμένου να επιτρέπεται η αναγνώριση πρόθεσης των εντολών του χρήστη.

\noindent\textbf{User Priority (5/5):} Η απαίτηση αυτή είναι πολύ σημαντική για τον χρήστη, καθώς ο καθρέφτης πρέπει να αναγνωρίζει τις προθέσεις του και να εκτελεί τις κατάλληλες ενέργειες.

\noindent\textbf{Technical Priority (5/5):} Η απαίτηση αυτή είναι πολύ σημαντική για το σύστημα, καθώς πρέπει να μπορεί να αναγνωρίζει τις προθέσεις του χρήστη για να εκτέλεση των επιθυμητών ενεργειών.

\subsubsection{\underline{ΜΛΑ-6}}
\noindent Το σύστημα πρέπει να λειτουργεί μόνο όταν ανιχνεύει πρόσωπο ανθρώπου.

\noindent\textbf{Περιγραφή:} Το σύστημα πρέπει να ανιχνεύει την ύπαρξη ανθρώπου μπροστά στον καθρέφτη και να σταματά τη λειτουργία του όταν δεν βρίσκει τίποτα.

\noindent\textbf{User Priority (4/5):} Η απαίτηση αυτή είναι σημαντική για τον χρήστη, καθώς δεν επιθυμεί ο καθρέφτης να ανιχνεύει τα λόγια του σε ανύποπτο χρόνο και να πιάνει εντολές χωρίς πρόθεση.

\noindent\textbf{Technical Priority (5/5):} Η απαίτηση αυτήν είναι πολύ σημαντική για το σύστημα, καθώς θα πρέπει να ελέγχει συνεχώς την ύπαρξη χρήστη μπροστά στον καθρέφτη και να σταματά τη λειτουργία του. Η απαίτηση αυτή θα βοηθήσει και στην εξοικονόμηση πόρων του συστήματος.

\subsubsection{\underline{ΜΛΑ-7}}
\noindent Το σύστημα πρέπει να συνδέεται στο διαδίκτυο με ικανοποιητικό εύρος σύνδεσης.

\noindent\textbf{Περιγραφή:} Το σύστημα πρέπει να υποστηρίζει σύνδεση στο διαδίκτυο προκειμένου να μπορεί να αλληλεπιδρά με τα διαφορετικά APIs και να λειτουργεί ορθά. Επομένως πρέπει να υπάρχει η απαραίτητη υποστήριξη σύνδεσης στο διαδίκτυο από το υλικό του συστήματος.

\noindent\textbf{User Priority (5/5):} Η απαίτηση αυτή είναι πολύ σημαντική για τον χρήστη, καθώς χωρίς σύνδεση διαδικτύου ο καθρέφτης δεν θα είναι σε θέση να εκτελέσει τις εφαρμογές του και να αναγνωρίσει τις εντολές του χρήστη.

\noindent\textbf{Technical Priority (5/5):} Η απαίτηση αυτή είναι πολύ σημαντική για το σύστημα, καθώς χρειάζεται σύνδεση στο διαδίκτυο για να λειτουργήσει ο καθρέφτης τις εφαρμογές του. 