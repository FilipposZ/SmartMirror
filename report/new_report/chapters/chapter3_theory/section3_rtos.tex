\section{Λειτουργικά Συστήματα Πραγματικού χρόνου}
\label{sec:theory_rtos}

Στην παρούσα εργασία, αναπτύχθηκε μία DSL, για ένα συγκεκριμένο λειτουργικό σύστημα πραγματικού χρόνου, το RIOT\footnote{\url{https://www.riot-os.org/}}.

Real-time σύστημα είναι εκείνο το οποίο εκτελεί τις δραστηριότητες που του ανατίθενται μέσα σε συγκεκριμένα χρονικά περιθώρια. Βασικά χαρακτηριστικά των RTOS είναι τα ακόλουθα:

\begin{itemize}
	\item Χρονοπρογραμματισμός
	\item Διαχείριση πόρων
	\item Συγχρονισμός
	\item Επικοινωνία
	\item Ακριβής χρονισμός
\end{itemize}
	
Παλαιότερα ένα RTOS αποσκοπούσε στην υλοποίηση μιας πολύ συγκεκριμένης λειτουργίας. Πλέον έχουν εξελιχθεί σε επίπεδο τέτοιο, ώστε να υποστηρίζουν λειτουργικά συστήματα πιο γενικού σκοπού, μέχρι και soft συστήματα πραγματικού χρόνου (δηλαδή συστήματα που λειτουργούν με υποβαθμισμένη απόδοση αν τα αποτελέσματα δεν παράγονται σύμφωνα με τις καθορισμένες χρονικές απαιτήσεις, σε αντίθεση με τα hard, τα οποία σε αυτήν την περίπτωση λειτουργούν λανθασμένα).

Τα RTOS δίνουν έμφαση στην προβλεψιμότητα, την αποτελεσματικότητα και περιλαμβάνουν δυνατότητες ώστε να υποστηρίζουν χρονικούς περιορισμούς. Υπάρχουν αρκετές γενικές κατηγορίες RTOS, όπως είναι οι πυρήνες (είτε είναι εμπορικοί είτε όχι), επεκτάσεις σε ήδη υπάρχοντα εμπορικά RTOS (π.χ. Unix, Linux), πυρήνες βασισμένους σε εξαρτήματα κ.α. \cite{bib:stankovic_2004}