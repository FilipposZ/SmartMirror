\section{Απαιτήσεις Συστήματος}
\label{section:pose_system_requirements}

\subsubsection{\underline{ΛΑ-1}}
\noindent Το σύστημα πρέπει να μπορεί να εκτιμάει την ανθρώπινη πόζα.

\noindent\textbf{Περιγραφή:} Το σύστημα πρέπει να κάνει προβλέψεις για την θέση των σημείων κλειδιών του ανθρώπινου σώματος έχοντας ως είσοδο τις εικόνες καρέ από την κάμερα του χρήστη ή από βίντεο εισόδου.

\noindent\textbf{User Priority (5/5):} Η απαίτηση εκτίμησης πόζας είναι απαραίτητη για τον χρήστη καθώς όλες οι λειτουργίες χρήσιμες για αυτόν βασίζονται στην δυνατότητα εκτίμησης πόζας.

\noindent\textbf{Technical Priority (5/5):} Η απαίτηση εκτίμησης πόζας είναι απαραίτητη για την εφαρμογή καθώς αποτελεί τον ακρογωνιαίο λίθο της λειτουργίας της.

\subsubsection{\underline{ΛΑ-2}}
\noindent Ο χρήστης πρέπει να μπορεί να εισάγει ορθές ασκήσεις.

\noindent\textbf{Περιγραφή:} Ο χρήστης πρέπει να μπορεί να ανεβάσει ένα βίντεο με μία επανάληψη άσκησης η οποία χρησιμοποιείται ως ορθή άσκηση αναφοράς. Τότε, η εφαρμογή εκτιμάει την ανθρώπινη πόζα από το βίντεο και αποθηκεύει τα εξαγόμενα δεδομένα για χρήση κατά τη λειτουργία \textsl{παιχνιδιού}.

\noindent\textbf{User Priority (3/5):} Η απαίτηση αυτή δεν είναι απαραίτητη για τον χρήστη, καθώς ικανοποιεί χρήστες που επιθυμούν την εξατομίκευση των ασκήσεων τους. Ο απλός χρήστης της εφαρμογής μπορεί να χρησιμοποιεί τα ήδη διαθέσιμα δεδομένα ορθών ασκήσεων, χωρίς να εισάγει ποτέ δικά του.

\noindent\textbf{Technical Priority (5/5):} Η απαίτηση της εισαγωγής ορθών ασκήσεων είναι άκρως απαραίτητη για την λειτουργία της εφαρμογής καθώς τα δεδομένα ορθής άσκησης χρησιμοποιούνται ως αναφορά για την λειτουργία \textsl{παιχνιδιού} της εφαρμογής.

\subsubsection{\underline{ΛΑ-3}}
\noindent Το σύστημα πρέπει να μπορεί να συγκρίνει την εκτέλεση άσκησης του χρήστη με κάποια επιλεγμένη άσκηση αναφοράς.

\noindent\textbf{Περιγραφή:} Το σύστημα πρέπει να μπορεί να εκτιμάει την πόζα του χρήστη σε κάθε καρέ και να την συγκρίνει με την πόζα που θα έπρεπε να έχει για αυτό το καρέ με βάση την επιλεγμένη άσκηση αναφοράς.

\noindent\textbf{User Priority (5/5):} Η απαίτηση σύγκρισης της πόζας του χρήστη με την πόζα αναφοράς είναι απαραίτητη για τον χρήστη καθώς η διαδικασία αυτή βρίσκεται στο επίκεντρο της λειτουργίας \textsl{παιχνιδιού} η οποία αποτελεί τον βασικό κόμβο σταθμό για τον χρήστη.

\noindent\textbf{Technical Priority (4/5):} Η απαίτηση σύγκρισης πόζας είναι απαραίτητη για την λειτουργία \textsl{παιχνιδιού} και εξ ακολούθως άκρως σημαντική για την εφαρμογή.

\subsubsection{\underline{ΛΑ-4}}
\noindent Το σύστημα πρέπει να προβάλει ταυτόχρονα τις εκτιμήσεις πόζας του χρήστη και της άσκησης αναφοράς.

\noindent\textbf{Περιγραφή:} Το σύστημα πρέπει σε κάθε καρέ να εμφανίζει στην οθόνη ταυτόχρονα τις εκτιμήσεις πόζας του χρήστη και της άσκησης αναφοράς.

\noindent\textbf{User Priority (5/5):} Η απαίτηση ταυτόχρονης προβολής είναι απαραίτητη για τον χρήστη καθώς με αυτό τον τρόπο μπορεί να δει τη πόζα αναφοράς που θα έπρεπε να έχει και την πόζα που έχει στην πραγματικότητα.

\noindent\textbf{Technical Priority (2/5):} Η απαίτηση αυτή δεν είναι απαραίτητη για την λειτουργία της εφαρμογής, αλλά αποτελεί προϋπόθεση άλλων απαιτήσεων, σημαντικών για την εκπλήρωση των στόχων της εφαρμογής.

\subsubsection{\underline{ΛΑ-5}}
\noindent Το σύστημα πρέπει να παρέχει οπτική ανατροφοδότηση στον χρήστη κατά την εκτέλεση της άσκησης σχετικά με τον τρόπο εκτέλεσής της.

\noindent\textbf{Περιγραφή:} Το σύστημα πρέπει να προβάλει στην οθόνη δυναμικά διορθωτικά βέλη, τα οποία ξεκινάνε από την προβαλλόμενη εκτίμηση πόζας του χρήστη και δείχνουν προς την πόζα αναφοράς. Έτσι, υποδεικνύεται σε κάθε καρέ η απόκλιση της πόζας του χρήστη από την πόζα αναφοράς.

\noindent\textbf{User Priority (5/5):} Η απαίτηση αυτή είναι απαραίτητη για τον χρήστη καθώς αποτελεί σημαντική κατευθυντήρια δύναμη, παρέχοντας ξεκάθαρη και άμεση ανατροφοδότηση για την πιο σωστή εκτέλεση της άσκησης αλλά και λειτουργώντας ως παρακινητής του χρήστη κατά την λειτουργία του \textsl{παιχνιδιού}. 

\noindent\textbf{Technical Priority (1/5):} Η απαίτηση αυτή δεν επηρεάζει με κανέναν τρόπο την λειτουργία της εφαρμογής.

\subsubsection{\underline{ΛΑ-6}}
\noindent Το σύστημα πρέπει να μετράει τον αριθμό επαναλήψεων που έγιναν και να τον προβάλει.

\noindent\textbf{Περιγραφή:} Το σύστημα πρέπει να προβάλει έναν μετρητή όπου φαίνεται ο αριθμός των επαναλήψεων που έχουν ολοκληρωθεί και που απομένουν για την συγκεκριμένη άσκηση. Με το πέρας της τελευταίας επανάληψης η άσκηση τερματίζεται.

\noindent\textbf{User Priority (5/5):} Η απαίτηση αυτή είναι άκρως σημαντική για τον χρήστη καθώς πρέπει να ξέρει σε πιο στάδιο εκτέλεσης της άσκησης βρίσκεται, λειτουργώντας επίσης ως παρακινητής για την ολοκλήρωση της άσκησης. 

\noindent\textbf{Technical Priority (4/5):} Η απαίτηση αυτή είναι σημαντική για την λειτουργία της εφαρμογής καθώς με την μέτρηση των επαναλήψεων που έχουν γίνει είναι δυνατή η λήξη της εκάστοτε άσκησης.

\subsubsection{\underline{ΛΑ-7}}
\noindent Ο χρήστης πρέπει να μπορεί να αναπαράγει τις ασκήσεις αναφοράς και να ελέγχει την αναπαραγωγή.

\noindent\textbf{Περιγραφή:} Ο χρήστης πρέπει να μπορεί να δει τα διαδοχικά καρέ της άσκησης αναφοράς. Επίσης, πρέπει να μπορεί να πατήσει παύση και συνέχεια της άσκησης καθώς και να διαλέξει το σημείο αναπαραγωγής με χρήση ενός ολισθητή.

\noindent\textbf{User Priority (3/5):} Ο απλός χρήστης μπορεί έτσι να μελετήσει τον τρόπο εκτέλεσης της εκάστοτε άσκησης. Στην περίπτωση εισαγωγής άσκησης αναφοράς η απαίτηση χρησιμεύει προς επιβεβαίωση των εκτιμώμενων δεδομένων αναφοράς.

\noindent\textbf{Technical Priority (2/5):} Η δυνατότητα αναπαραγωγής των ασκήσεων αναφοράς δεν είναι απαραίτητη για την λειτουργία της εφαρμογή.

\subsubsection{\underline{ΛΑ-8}}
\noindent Ο χρήστης πρέπει να μπορεί να επεξεργάζεται τα δεδομένα των ασκήσεων αναφοράς.

\noindent\textbf{Περιγραφή:} Τα δεδομένα των ασκήσεων αναφοράς ενδέχεται να περιέχουν σφάλματα εκτίμησης. Κατά συνέπεια, ο χρήστης πρέπει να μπορεί να επεξεργάζεται τις ασκήσεις αναφοράς ώστε να ελαχιστοποιήσει το σφάλμα. Για κάθε σημείο κλειδί του ανθρώπινου σώματος και για κάθε άξονα περιστροφής ο χρήστης μπορεί να ορίσει μία σταθερή τιμή ή ένα περιορισμένο εύρος για τις μοίρες περιστροφής του σημείου κλειδιού γύρω από τον εκάστοτε άξονα.

\noindent\textbf{User Priority (2/5):} Η απαίτηση της επεξεργασίας των ορθών ασκήσεων δεν είναι απαραίτητη για τον χρήστη, καθώς η εφαρμογή λειτουργεί ως αρωγός στην παρακίνηση του και η απόλυτη ακρίβεια δεν είναι ρεαλιστικά εφικτή.

\noindent\textbf{Technical Priority (2/5):} Η απαίτηση αυτή δεν είναι απαραίτητη για την σωστή λειτουργία της εφαρμογής, εφόσον τα βίντεο ορθών ασκήσεων είναι καλής ποιότητας ώστε οι εκτιμήσεις ασκήσεων αναφοράς να έχουν ελάχιστα σφάλματα. 

\subsubsection{\underline{ΛΑ-9}}
\noindent Ο χρήστης πρέπει να μπορεί να αλλάξει το χρώμα του προβαλλόμενου ανθρώπινου μοντέλου.

\noindent\textbf{Περιγραφή:} Ο χρήστης πρέπει να μπορεί να αλλάξει το χρώμα του ανθρώπινου μοντέλου ώστε να ικανοποιεί την αισθητική του.

\noindent\textbf{User Priority (3/5):} Η απαίτηση της αλλαγής χρώματος είναι σημαντική για τον χρήστη καθώς η ικανοποίηση της αισθητικής του είναι απαραίτητη για την παρακίνηση του.

\noindent\textbf{Technical Priority (1/5):} Η απαίτηση αυτή δεν επηρεάζει με κανέναν τρόπο τη λειτουργία της εφαρμογής.




\noindent\subsection{Μη Λειτουργικές Απαιτήσεις}

\subsubsection{\underline{ΜΛΑ-1}}
\noindent Το σύστημα πρέπει να προβάλει το ανθρώπινο μοντέλο με τουλάχιστον 30 καρέ το δευτερόλεπτο.

\noindent\textbf{Περιγραφή:} Το σύστημα, στην λειτουργία \textsl{παιχνιδιού}, πρέπει να εκτιμάει την ανθρώπινη πόζα, να κάνει την απαραίτητη επεξεργασία των δεδομένων και να τα προβάλει αρκετά γρήγορα ώστε να επιτυγχάνει 30 καρέ το δευτερόλεπτο, δηλαδή ο συνολικός χρόνος επεξεργασίας του καρέ εισόδου εώς την προβολή του ανθρώπινου μοντέλου να είναι λιγότερο από 35ms.

\noindent\textbf{User Priority (5/5):} Η απαίτηση αυτή είναι άκρως σημαντική για τον χρήστη καθώς η ομαλή διαδοχή των καρέ στο μάτι είναι απαραίτητη για την ικανοποιητική χρήση της εφαρμογής.

\noindent\textbf{Technical Priority (5/5):} Η απαίτηση αυτή είναι πολύ σημαντική για το σύστημα, καθώς καθορίζει τα τεχνικά χαρακτηριστικά και την δομή του, την επιλογή μοντέλων για την εκτίμηση πόζας και τις απαιτήσεις των αλγορίθμων απεικόνισης των δεδομένων.

\subsubsection{\underline{ΜΛΑ-2}}
\noindent Το σύστημα πρέπει να μπορεί να προσαρτηθεί στο λειτουργικό του καθρέφτη.

\noindent\textbf{Περιγραφή:} Το σύστημα πρέπει να μπορεί να προσαρτηθεί εύκολα στο λειτουργικό σύστημα.

\noindent\textbf{User Priority (5/5):} Η απαίτηση αυτή είναι άκρως σημαντική για τον χρήστη, καθώς η διεπαφή του με το σύστημα γίνεται μέσω του λειτουργικού του καθρέφτη.

\noindent\textbf{Technical Priority (5/5):} Η απαίτηση αυτή είναι άκρως σημαντική για το σύστημα, καθώς η εφαρμογή λειτουργεί στα πλαίσια του λειτουργικού του καθρέφτη.


\subsubsection{\underline{ΜΛΑ-3}}
\noindent Το σύστημα πρέπει να παρέχει στο λειτουργικό του καθρέφτη τις απαραίτητες φωνητικές εντολές για την λειτουργία του.

\noindent\textbf{Περιγραφή:} Το σύστημα πρέπει να εξάγει μία διεπαφή προς το λειτουργικό του καθρέφτη, όπου καθορίζονται οι απαιτούμενες φωνητικές εντολές και οι λειτουργίες που επιτελούν.

\noindent\textbf{User Priority (5/5):} Η απαίτηση αυτή είναι άκρως σημαντική για τον χρήστη, καθώς έτσι επιτυγχάνεται η αλληλεπίδρασή του με το σύστημα με την χρήση φωνής.

\noindent\textbf{Technical Priority (5/5):} Η απαίτηση αυτή είναι άκρως σημαντική για το σύστημα, καθώς με αυτόν τον τρόπο γίνεται ο έλεγχος των λειτουργιών του συστήματος.


\subsubsection{\underline{ΜΛΑ-4}}
\noindent Το σύστημα πρέπει να είναι αρθρωτό.

\noindent\textbf{Περιγραφή:} Οι αλγόριθμοι που χρησιμοποιεί το σύστημα, όπως ο αλγόριθμος εκτίμησης πόζας και ο αλγόριθμος απεικόνισης του ανθρώπινου μοντέλου, πρέπει να μπορούν να χρησιμοποιηθούν αυτόνομα. Με αυτόν τον τρόπο, η αξιοποίηση τους για ανάπτυξη επιπλέον εφαρμογών για τον καθρέφτη διευκολύνεται σε μεγάλο βαθμό.

\noindent\textbf{User Priority (2/5):} Η απαίτηση αυτή δεν επηρεάζει τον χρήστη κατά την λειτουργία της εφαρμογής, αλλά αυξάνει την ευκολία ανάπτυξης αντίστοιχων εφαρμογών, οι οποίες επαυξάνουν τις δυνατότητες του έξυπνου καθρέφτη.

\noindent\textbf{Technical Priority (4/5):} Η απαίτηση αυτή είναι πολύ σημαντική για το σύστημα, καθώς καθιστά την λειτουργία του συστήματος πιο εύρωστη, την ανάπτυξη του πιο εύκολη αλλά και προωθεί την επεκτασιμότητα των εφαρμογών του καθρέφτη.

\subsubsection{\underline{ΜΛΑ-5}}
\noindent Το σύστημα πρέπει να είναι διαισθητικά απλό και εύχρηστο.

\noindent\textbf{Περιγραφή:} Το σύστημα πρέπει να είναι απλό στην χρήση του και να παρέχει ενημερωτικό κείμενο σχετικά με την κατάσταση του.

\noindent\textbf{User Priority (5/5):} Η απαίτηση αυτή είναι άκρως σημαντική για τον χρήστη, καθώς η εύχρηστη λειτουργία της εφαρμογής εξωραΐζει την εμπειρία του χρήστη με το σύστημα.

\noindent\textbf{Technical Priority (3/5):} Η απαίτηση αυτή δεν επηρεάζει την λειτουργία του συστήματος, εντούτοις η απλοϊκότητα των λειτουργιών συνεπάγεται και απλοϊκότητα υλοποίησης, βελτιώνοντας την ποιότητα και την ευρωστία του συστήματος.